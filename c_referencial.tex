% ---
% Capitulo de revisão de literatura
% ---


\chapter{Referencial Teórico}\label{ch:referencial_teorico}


\section{Autenticação e Identidade}\label{sec:autenticacaoeidentidade}
O problema de estabelecer uma associação entre um indivíduo e uma identidade pode ser dividido em duas categorias: autenticação e identificação~\cite{magalhaes2003biometria}.

\subsection{Autenticação}\label{subsec:autenticacao}
É o processo que verifica a autenticidade de um usuário, processo ou dispositivo.
Essa verificação confirma a legitimidade da entidade em questão.
Durante a autenticação, a parte que examina assegura que a entidade sendo verificada é genuína, e esta última participa ativamente na troca de informações~\cite{usmonov2021identification}.


Para~\cite{conti2017biometric} existem três tipos de autenticação\label{tipos-autenticacao}:

\begin{itemize}
    \item \textbf{Baseada em conhecimento:} utiliza informações que a pessoa sabe, como uma senha.
    \item \textbf{Com base em algo que a pessoa possua}, como um \textit{token} ou cartão inteligente.
    \item \textbf{Com base em características físicas da pessoa}, também conhecidas como biometria.
\end{itemize}


\section{Biometria}\label{sec:biometria}
O termo \("\)biometria\("\) vem das palavras gregas \("\)bios\("\) (vida) e \("\)metrikos\("\) (medida)~\cite{magalhaes2003biometria}.
É o estudo que visa identificar um indivíduo com base em suas características fisiológicas e comportamentais~\cite{handa2019comparative}, ou seja, biometria é uma forma de identificar pessoas usando características físicas ou comportamentais únicas.

Dos tipos mencionados em~\ref{tipos-autenticacao}, a biometria é tida como a abordagem mais segura, já que os atributos físicos de uma pessoa não podem ser furtados, cedidos ou esquecidos.
Falsear a autenticação biométrica é complicado e inviável, uma vez que avalia características singulares do indivíduo~\cite{dos2019tecnologias}.

\subsection{Tecnologias Biométricas}\label{subsec:biometria-tecnologias}

\subsubsection{Reconhecimento facial}\label{subsubsec:reconhecimento-facial}
É uma tecnologia capaz de identificar uma pessoa a partir de uma imagem digital ou de um vídeo, de modo que é comparado as características faciais selecionadas de uma determinada imagem com os rostos existentes num banco de dados~\cite{orvalho2019reconhecimento}.

\begin{longtable}[c]{|l|l|}
    \hline
    \multicolumn{1}{|c|}{\textbf{Problemas}} & \multicolumn{1}{c|}{\textbf{Reconhecimento Facial}} \\ \hline
    \endfirsthead
    \endhead
    Sensor & \begin{tabular}[c]{@{}l@{}}
                 Resolução espacial, taxa\\ de quadros, distância da\\ câmera.
    \end{tabular} \\ \hline
    Envelhecimento & \begin{tabular}[c]{@{}l@{}}
                         Mudanças geométricas\\ entre a infância e a\\ adolescência, rugas e\\ flacidez facial.
    \end{tabular} \\ \hline
    Interação com o usuário & Poses e expressões. \\ \hline
    \begin{tabular}[c]{@{}l@{}}
        Mudanças no meio\\ ambiente
    \end{tabular} & \begin{tabular}[c]{@{}l@{}}
                        iluminação e cena de\\ fundo.
    \end{tabular} \\ \hline
    Outros fatores & \begin{tabular}[c]{@{}l@{}}
                         Maquiagem, acessórios,\\ e oclusão
    \end{tabular} \\ \hline
    \caption*{Fonte: Adaptado de~\cite{usmonov2021identification}}
\end{longtable}



\section{Localização Exterior - GPS}\label{sec:localizacao}
GPS é a sigla da abreviatura de \textit{Global Positioning System}, ou Sistema de Posicionamento Global em português~\cite{gpsdesigning}.
Também chamado de NAVSTAR-GPS (Sistema de Navegação por Satélites com Tempo e Distância), é um sistema de navegação por rádio criado pelo Departamento de Defesa dos EUA. Foi criado originalmente para a navegação militar americana~\cite{novais2014localizaccao}.

É o método de identificar a posição geográfica de um objeto, pessoa ou dispositivo eletrônico.
Para isso, utiliza-se de informações provenientes de sinais de \hypertarget{receptores}{ \textit{GPS}, torres de celular, endereços \textit{IP} ou redes  \textit{Wi-Fi}}.
Em resumo, permite o rastreamento em tempo real da localização de uma entidade~\cite{da2019sistemas}.

O GPS consiste em uma constelação de 24 satélites que orbitam a Terra a uma altitude aproximada de 20.200 km acima do nível do mar.
Essa configuração permite que os~\hyperlink{receptores}{receptores} determinem sua posição em qualquer lugar do planeta com notável precisão~\cite{el2002introduction}.

\subsection{Localização Interna - (Indoor)}\label{subsec:localizacao-indoor}

É uma tecnologia projetada para localização em ambientes internos, como andares de prédios, tuneis, salas ou auditórios~\cite{mittelstadt2018bluepath}.
Um dos fatores que aumentam a confiabilidade da localização de uma pessoa ou objeto em um determinado local é o GPS, entretanto essa tecnologia apresenta melhor desempenho em ambientes abertos, tendo uma grande imprecisão em ambientes fechados.

Quando um receptor está em um espaço interno, torna-se muito difícil decodificar os sinais GPS, uma vez que eles são atenuados por edifícios e paredes.
Isso resulta em perda de potência do sinal e, consequentemente, em erros na localização do receptor.
Nesse contexto, existe a localização indoor~\cite{mittelstadt2018bluepath}.

\subsubsection{Tecnologias de Localização Indoor}\label{subsubsec:tecnologias-localizacao-indoor}
Conforme o entedimento de~\cite{novais2014localizaccao} em sua tese de mestrado \("\)Localização Indoor em Ambientes Inteligentes\("\), ele que explica que as tecnologias de localização indoor geralmente são sem fio, visando uma maior aceitação dos usuários, pois andar com cabos para se localizar em um ambiente não é conveniente.
Dessa forma, as tecnologias wireless proporcionam mais conforto, comodidade e segurança.
Há várias maneiras de prover a localização em ambientes internos, algumas serão explicadas a aseguir conforme o autor.

\subsubsubsection{Wi-Fi}\label{subsubsubsec:wifi}
A Wi-Fi (Wireless Fidelity) foi estabelecida em 1999 pela Interbrand\footnote{\url{https://interbrand.com/}} para a Wi-Fi Alliance\footnote{\url{https://www.wi-fi.org/}}, que visa assegurar a compatibilidade entre dispositivos Wi-Fi. Utiliza a norma IEEE 802.11, com acréscimos como os padrões 802.11b/g/n/ac, e permite comunicação entre dispositivos através de ondas de rádio acima de 2.4 GHz. Os pontos de acesso conectam vários dispositivos sem fio e podem comunicar entre si, com alcance de aproximadamente 35 metros em ambientes internos e 110 metros em externos.
A prevalência de redes Wi-Fi em locais públicos e privados facilitou sua adoção em sistemas de localização indoor.
Esses sistemas são específicos para cada edifício e podem guiar pessoas ou robôs com dispositivos móveis Wi-Fi. A precisão aumenta ao limitar as possíveis posições dos dispositivos e fornecer plantas dos locais para reduzir distorções de sinal.
Os sistemas de localização podem operar nos próprios dispositivos (localização implícita) ou em servidores (localização explícita).
A configuração envolve uma fase de treino, coletando intensidade do sinal Wi-Fi e identificadores SSID, e uma fase online que compara os dados recebidos com os mapas de rádio para determinar a posição do dispositivo via técnicas de localização determinísticas ou probabilísticas.

\subsubsubsection{ZigBee}\label{subsubsubsec:zigbee}
ZigBee é um conjunto de protocolos de comunicação de baixa taxa de dados para conexão sem fio de curto alcance, baseado no padrão \textit{IEEE 802.15.4} (É um padrão para redes sem fio de baixo consumo de energia e baixa velocidade de dados, usado em \textit{IoT}) e desenvolvido pela Connectivity Standards Alliance (CSA) \footnote{\url{https://csa-iot.org/}} (anteriormente conhecida como ZigBee Alliance), uma organização sem fins lucrativos.
A tecnologia é econômica e de baixo consumo energético, ideal para automação residencial e monitoramento, mas também é aplicável em localização indoor.
Funciona com rádio frequência e requer a posição conhecida de emissores no ambiente para localizar objetos ou dispositivos através da comunicação entre os emissores.

\subsubsubsection{RFID}\label{subsubsubsec:rfid}
A tecnologia RFID (Radio-Frequency Identification) é usada para armazenar e recuperar dados por meio de transmissão eletromagnética.
O RFID permite identificar e rastrear itens e foi precursor dos sistemas modernos ao identificar aviões amigos ou inimigos.
Um sistema RFID inclui leitores e etiquetas, que comunicam usando frequência de rádio.
Etiquetas passivas não têm bateria e refletem o sinal do leitor com informações moduladas, enquanto etiquetas ativas têm alcance maior, até 100 metros, e são mais rápidas, sendo lidas em menos de 100 milissegundos.
Para localização indoor, RFID é mais complexo, geralmente utilizando o dispositivo móvel como leitor e etiquetas distribuídas como pontos de referência para estimar a localização através do sinal de força recebido (RSSI), exigindo que o edifício seja dividido em zonas para essa finalidade.