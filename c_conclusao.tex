\chapter{Conclusão}\label{cap:conclusao}

O presente trabalho teve como objetivo principal realizar um estudo sobre aplicações móveis construídas a partir de tecnologias de desenvolvimento híbrido e aplicar estes conhecimentos no desenvolvimento de uma aplicação voltada à área de Segurança Pública, direcionada ao atendimento humanizado e padronizado de pessoas em situação de vulnerabilidade.

Com os objetivos e funcionalidades definidos, iniciou-se a implementação da aplicação proposta, a qual ainda está em processo de desenvolvimento no momento da entrega deste trabalho, pois nem todas as funcionalidades foram implementadas com sucesso. Como produtos finais gerados além do estudo das tecnologias, foi realizado o projeto das funcionalidades, que envolvem os diagramas de classes e de caso de uso, assim como o protótipo das telas. Também foram apresentadas aqui as telas do aplicativo desenvolvidas com a tecnologia Flutter e ilustram a forma como o aplicativo consume informações da aplicação \textit{web}.  

A partir do estudo, foi possível perceber que as tecnologias computacionais evoluem de maneira rápida e eficaz, pois cada novo passo bem sucedido de uma aplicação se torna um fator de decadência de outras tecnologias, que podem passar a ser obsoletas rapidamente. Esta evolução é necessária tendo em vista que a população mundial está em constante conexão com o mundo virtual e os sistemas precisam estar sempre em estado de integração e compatibilidade.