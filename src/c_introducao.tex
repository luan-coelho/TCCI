% ----------------------------------------------------------
% Introdução (exemplo de capítulo sem numeração, mas presente no Sumário)
% ----------------------------------------------------------

\chapter{Introdução}\label{intro}

Ao passar dos anos, os aparelhos computacionais evoluíram e, com eles, a maneira de se desenvolver programas de computador. Seguindo essa evolução, aplicações \textit{desktop} perderam espaço para sistemas \textit{web}, e hoje em dia, o principal interesse do mercado é o desenvolvimento para a telefonia móvel, pelo simples fato de ser amplamente utilizado pela população \cite{leite2017comparativo}.

O desenvolvimento de software para dispositivos móveis deve ser pensado de acordo com o seu SO (Sistema Operacional), sendo os sistemas Android  e iOS  os mais utilizados atualmente. Desta maneira, para facilitar a unificação do desenvolvimento para diferentes SO's, descartando-se a necessidade de manter duas equipes de programação para dar suporte às duas abordagens , algumas empresas de tecnologia criaram frameworks de desenvolvimento para atender as duas plataformas.

Atualmente existem duas principais tecnologias de desenvolvimento híbrido que disputam entre si, são elas React-Native e Flutter. No presente trabalho ambas as tecnologias foram estudadas e, por critérios pessoais de desenvolvimento, o framework Flutter foi escolhido para a construção da aplicação.

Flutter utiliza a linguagem Dart em seu desenvolvimento e para desenhar os compomemtes visuais da aplicalção, é utilizada a ferramenta Skia 2D , uma biblioteca gráfica poderosa compatível com diversos tipos de dispositivos computacionais, e a junção destes fatores acarreta em um desempenho de boa qualidade e que vem atraindo desenvolvedores \cite{zammetti2019practical}.

\section{Objetivos}
\subsection{Objetivo Geral }

Propor um padrão de processo de desenvolvimento para assistência farmacêutica, objetivando contribuir para a melhoria da qualidade dos serviços, promover a padronização dos processos e estimular a adoção de boas práticas em assistência farmacêutica.

\subsection{Objetivos Específicos}
\begin{itemize}

\item Realizar revisão bibliográfica sobre as metodologias de desenvolvimento de processos para assistência farmacêutica.
\item Identificar as principais necessidades e desafios relacionados à assistência farmacêutica.
\item Definir as etapas e atividades do processo de desenvolvimento proposto.

\end{itemize}
