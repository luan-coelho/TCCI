\chapter{Metodologia}\label{ch:metodologia}
No capítulo em questão, os procedimentos metodológicos destinados à pesquisa e ao desenvolvimento subsequente da aplicação serão expostos.
Inicialmente, haverá uma contextualização acerca do tema abordado, seguida pela descrição do tipo de pesquisa a ser implementada no projeto e pelos materiais que serão empregados.

\citeonline{gerhardt2009metodos} \cite[pp. 10--15]{gerhardt2009metodos} abordam o significado da metodologia, enfatizando seu papel na organização dos caminhos e instrumentos necessários ao longo da pesquisa.
Eles também diferenciam os termos \("\)metodologia\("\) e \("\)métodos\("\), salientando que a metodologia se interessa pela validade dos caminhos adotados para alcançar os objetivos da pesquisa, enquanto os métodos referem-se aos procedimentos empregados no decorrer da investigação científica.


\section{Palavras-chave}\label{sec:palavras-chave}
Nesta seção, descrevemos as palavras-chave utilizados na busca bibliográfica para este trabalho.

\begin{itemize}
    \item Autenticidade de frequência
    \item Biometria e Autenticação
    \item Tecnologias Biométricas
    \item Reconhecimento Facial
    \item Localização Indoor
    \item Aplicativo móvel
\end{itemize}


\section{Fonte de Dados}\label{sec:fonte-dados}
\begin{itemize}
    \item Google Scholar
    \item Periódicos CAPES
\end{itemize}


\section{Critérios de seleção}\label{sec:criterios-de-inclusao}
Nesta seção, descrevemos os critérios de inclusão e exclusão utilizados na busca bibliográfica para este trabalho.

\subsection{Critérios de Inclusão}\label{subsec:criterios-de-inclusao}

\begin{enumerate}
    \item \textbf{Publicações a partir dos anos 2000:} Para o uso de biometria e principais tecnologias, bem como conceitos tópicos mais conceituais, publicações mais antigas se mostraram mais claras e diretas em suas abordagens.
    Para temas de tecnologia foram priorizados trabalhos publicados nos últimos 5 anos para assegurar a atualidade.
    \item \textbf{Relevância temática:} foram incluídos trabalhos relacionados ou que abordam diretamente o tema da pesquisa.
    \item \textbf{Tipo de publicação:} selecionaram-se artigos, teses, dissertações e livros.
    \item \textbf{Idioma:} foram consideradas publicações em português e inglês.
    \item \textbf{Disponibilidade de acesso:} apenas documentos com acesso integral foram contemplados.
\end{enumerate}

\subsection{Critérios de Exclusão}\label{subsec:criterios-de-exclusao}

\begin{enumerate}
    \item \textbf{Publicações muito antigas:} foram excluídos trabalhos publicados antes dos anos 2000.
    \item \textbf{Baixa relevância temática:} foram excluídos trabalhos que não se concentravam especificamente no tema de pesquisa.
    \item \textbf{Fontes não-acadêmicas:} foram descartados blogs, notícias e outras fontes não revisadas por pares, com exceção de leis.
    \item \textbf{Idioma inacessível:} foram excluídos trabalhos em idiomas que não eram compreensíveis ou traduzíveis.
    \item \textbf{Acesso restrito:} foram excluídos trabalhos que não estavam disponíveis para consulta completa e de forma gratuita.
    \item \textbf{Estudos de baixa qualidade metodológica:} foram descartados trabalhos com falhas metodológicas significativas.
\end{enumerate}


\section{Materiais Utilizados}\label{sec:materiais-utilizados}
Nessa seção será apresentado os materiais e ferramentas utilizadas para o desenvolvimento desta pesquisa:

\begin{table}[H]
    \centering
    \caption{Lista dos materiais utilizados}
    \label{tab:materiais-utilizados}
    \begin{tabular}{|c|c|c|}
        \hline
        & \textbf{Material} & \textbf{Versão} \\ \hline
        1 & IntelliJ IDEA     & 2023.2.5        \\ \hline
        2 & Dart              & 3.1             \\ \hline
        3 & Flutter           & 3.13            \\ \hline
        4 & Java              & 17              \\ \hline
        5 & Quarkus           & 3.5.0           \\ \hline
        6 & PostgreSQL        & 16.1            \\ \hline
        7 & Python            & 3.12            \\ \hline
        8 & Flask             & 3.0.0           \\ \hline
        9 & face\_recognition & 1.2.2           \\ \hline
    \end{tabular}
\end{table}

\begin{enumerate}
    \item \textit{IDE} utilizada no Desenvolvimento da aplicação e artefatos de software
    \item Linguagem de programação utilizada na construção da aplicação móvel
    \item \textit{Framework} Dart utilizado na construção da aplicação móvel
    \item Linguagem de programação utilizada na construção do \textit{Web Service}
    \item \textit{Framework} Java utilizado na construção do \textit{Web Service}
    \item \textit{SGDB} utilizado para armazenamento de dados da aplicação.
    \item Linguagem de programação utilizada na construção do \textit{microservice} de reconhecimento facial
    \item \textit{Framework} Python utilizado na construção do \textit{microservice} de reconhecimento facial
    \item Biblioteca Python utilizada no \textit{microservice} em Flask para verificação de reconhecimento facial
\end{enumerate}
