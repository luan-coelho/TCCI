\chapter{Metodologia}\label{ch:metodologia}


\section{Palavras-chave}\label{sec:palavras-chave}
Nesta seção, descrevemos as palavras-chave utilizados na busca bibliográfica para este trabalho.

\begin{itemize}
    \item Autenticidade de frequência
    \item Biometria e Autenticação
    \item Tecnologias Biométricas
    \item Reconhecimento Facial
    \item Localização Indoor
    \item Aplicativo móvel
\end{itemize}


\section{Fonte de Dados}\label{sec:fonte-dados}
\begin{itemize}
    \item Google Scholar
    \item Periódicos CAPES
\end{itemize}


\section{Critérios de seleção}\label{sec:criterios-de-inclusao}
Nesta seção, descrevemos os critérios de inclusão e exclusão utilizados na busca bibliográfica para este trabalho.

\subsection{Critérios de Inclusão}\label{subsec:criterios-de-inclusao}

\begin{itemize}
    \item Publicações a partir dos anos 2000: Para o uso de biometria e principais tecnologias, bem como conceitos tópicos mais conceituais, publicações mais antigas se mostraram mais claras e diretas em suas abordagens.
    Para temas de tecnologia foram priorizados trabalhos publicados nos últimos 5 anos para assegurar a atualidade.
    \item Relevância temática: foram incluídos trabalhos relacionados ou que abordam diretamente o tema da pesquisa.
    \item Tipo de publicação: selecionaram-se artigos, teses, dissertações e livros.
    \item Idioma: foram consideradas publicações em português e inglês.
    \item Disponibilidade de acesso: apenas documentos com acesso integral foram contemplados.
\end{itemize}

\subsection{Critérios de Exclusão}\label{subsec:criterios-de-exclusao}

\begin{itemize}
    \item Publicações muito antigas: foram excluídos trabalhos publicados antes dos anos 2000.
    \item Baixa relevância temática: foram excluídos trabalhos que não se concentravam especificamente no tema de pesquisa.
    \item Fontes não-acadêmicas: foram descartados blogs, notícias e outras fontes não revisadas por pares.
    \item Idioma inacessível: foram excluídos trabalhos em idiomas que não eram compreensíveis ou traduzíveis.
    \item Acesso restrito: foram excluídos trabalhos que não estavam disponíveis para consulta completa e de forma gratuita.
    \item Estudos de baixa qualidade metodológica: foram descartados trabalhos com falhas metodológicas significativas.
\end{itemize}


\section{Materiais Utilizados}\label{sec:materiais-utilizados}
Nessa seção será apresentado os materiais e ferramentas utilizadas nessa pesquisa:

\begin{table}[H]
    \centering
    \caption{Lista dos materiais utilizados}
    \label{tab:my-table}
    \begin{tabular}{|c|c|}
        \hline
        \textbf{Material} & \textbf{Versão} \\ \hline
        IntelliJ IDEA     & 2023.2.5        \\ \hline
        Dart              & 3.1             \\ \hline
        Flutter           & 3.13            \\ \hline
        Java              & 17              \\ \hline
        Quarkus           & 3.5.0           \\ \hline
        PostgreSQL        & 16.1            \\ \hline
        Python            & 3.12            \\ \hline
        Flask             & 3.0.0           \\ \hline
        face\_recognition & 1.2.2           \\ \hline
    \end{tabular}
\end{table}

\footnote{\url{https://www.jetbrains.com/idea/}}

Em termos de harware, utilizou-se dos seguintes materiais citados a seguir na
tabela.