% ---
% Capitulo de METODOLOGIA
% ---

\chapter{Metodologia}\label{ch:metodologia}

O presente trabalho é compreendido por uma pesquisa exploratória, tendo como base de estudo, a pesquisa bibliográfica e de documentação das tecnologias apresentadas.
A partir da definição do tema a ser estudado, fez-se a realização de revisão bibliográfica e documental utilizando-se da investigação em livros de diversos escritores, artigos e trabalhos publicados em instutuições de ensino, assim como a documentação de software oficial das aplicações abordadas.


\section{Materiais e Métodos}\label{sec:materiais-e-metodos}

O objetivo deste trabalho é entender como se dá o desenvolvimento de aplicações móveis e, para atingir este objetivo, foi preciso realizar pesquisas que dariam base para a construção da aplicação proposta.
Primeiramente, foi necessário estudar de forma geral todas as tecnologias envolvidas para a construção da aplicação para que fossem definidos os passos requeridos para atingir um produto final satisfatório.

Para construir uma aplicação móvel, é necessário definir em quais plataformas ela precisará estar disponível, também definir questões primordiais de segurança (como a necessidade de autenticação de usuário), determinar a dinâmica da aplicação, se requer conexão constante com a internet e também determinar o público alvo deste programa.
Nesta etapa, ocorreram reuniões para levantamento das funcionalidades do sistema e definição das tecnologias a serem utilizadas tendo em vista um prévio conhecimento de como as tecnologias que se pretendem utilizar funcionam e suas vantagens.

Com isso, definiu-se que o aplicativo para dispositivos móveis seria baseado em tecnologia híbrida, eliminando o problema de disponibilidade em sistemas operacionais distintos.
Caso contrário, desenvolver de forma nativa, acarretaria um maior tempo de desenvolvimento ou seria necessário alocar mais pessoas ao projeto.
O \textit{framework} de desenvolvimento de aplicações multiplataformas Flutter, foi escolhido para este projeto por estar em rápida ascensão e apresenta muitas vantagens.

O próximo passo executado foi a modelagem das funcionalidades da aplicação a partir do desenho das telas e a interação entre elas.
Para este propósito, foi utilizado o programa Adobe XD \footnote{https://www.adobe.com/br/products/xd/details.html} que auxilia no desenho das telas e na simulação de interação.

Portanto, com a tecnologia de desenvolvimento já definida, foi necessário preparar o ambiente de programação.
Foi utilizado um programa de desenvolvimento definido como padrão para aplicações feitas com Flutter, seu nome é Android Studio \footnote{https://developer.android.com/studio}, e provê diversas funcionalidades para auxílio ao desenvolvedor.
Para o SO utilizado no desenvolvimento desse projeto (Windows 10 \footnote{https://www.microsoft.com/pt-br/windows/}), foi necessário realizar a instalação do SDK Flutter.
Também foi utilizado o Gerenciador de Dispositios Virtuais Android \footnote{https://developer.android.com/studio/run/managing-avds.html?hl=pt-br}, que é um emulador, um programa que simula um dispositivo móvel com Android para facilitar a execução e teste dos aplicativos sem a necessidade de um dispositivo físico.





