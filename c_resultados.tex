\chapter{Resultados}\label{ch:resultados}
Neste capítulo, são apresentados os resultados obtidos durante a elaboração deste trabalho, tais como: a criação de artefatos de software, a descrição da arquitetura proposta e a implementação da ferramenta.
Por fim, são abordadas as análises obtidas por meio dos testes realizados.


\section{Arquitetura da aplicação}\label{sec:arquitetura-aplicacao}


\section{Diagramas}\label{sec:diagramas}
Esta seção se dedica a apresentar e detalhar, por meio de diagramas, o funcionamento do aplicativo a ser desenvolvido.
Ao todo, foram criados três diagramas: o diagrama de caso de uso, diagrama de classes e o diagrama de entidade-relacionamento.

\subsection{Diagrama de Caso de Uso}\label{subsec: diagrama-caso-uso}
A Figura 8 apresenta o diagrama de caso de uso, que descreve as principais
funcionalidades do sistema e como seus agentes interagem com o sistema.


\subsection{Diagrama de Classes}\label{subsec: diagrama-classes}
O diagrama de classes UML (do inglês, Unified Modeling Language) apresentado
abaixo (Figura 9) servirá como modelo das entidades presentes no sistema de modo a guiar
o desenvolvimento da aplicação, suas funcionalidades, e como cada classe se relaciona uma
com a outra.
É composto por treze classes e cinco enumeradores. Seguem as descrições de cada item do diagrama:

\section{Protótipo de Telas}\label{sec:prototipo}



